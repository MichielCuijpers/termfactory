
\documentclass{article}

\usepackage[T1]{fontenc}
\usepackage{geometry}
\usepackage{natbib}
\usepackage{graphicx}
\usepackage{amsmath}
\usepackage{physics}
\usepackage{simplewick}

\usepackage{color}
\NewDocumentCommand{\disconnected}{G{}}{\colorbox{yellow}{$#1$}}
\NewDocumentCommand{\red}{G{}}{{\color{red}#1}}
\NewDocumentCommand{\blue}{G{}}{{\color{blue}#1}}
\NewDocumentCommand{\magenta}{G{}}{{\color{magenta}#1}}

\geometry{legalpaper, landscape, left=0.4in, right=0.4in, top=0.25in, bottom=0.5in}

\newcommand{\bh}{\textbf{h}}
\newcommand{\bs}{\textbf{s}}
\newcommand{\bt}{\textbf{t}}
\newcommand{\bw}{\textbf{w}}
\newcommand{\bc}{\textbf{c}}
\newcommand{\bd}{\textbf{d}}

\newcommand{\up}[1]{\hat{#1}^{\dagger}}
\newcommand{\down}[1]{\hat{#1}}

\begin{document}

%
%
%
%
\section{Automated equations}
All equations are derived assuming a Hamiltonian and $\hat{S}$ of the form:
\begin{equation}
    \hat{H} = \sum_{a,b} \ket{a}\bra{b}h^{a}_{b}(q)
\qquad\qquad
    \hat{S} = \sum_{c,d} \ket{c}\bra{d}s^{c}_{d}(q)
\qquad\qquad
    \hat{\Omega} = \down{i} + \up{i} + \down{i}\down{j} + \down{i}\up{j} + \up{i}\up{j} + \cdots
\end{equation}
when we limit ourselves to at most 2nd order terms (creation operator: $\down{b}$, annihilation operator: $\up{b}$)
\begin{equation}
    \bh(q) = \bh + \bh_{i}\down{i} + \frac{1}{2}\bh_{ij}\down{i}\down{j}
\end{equation}
and
\begin{equation}
    \bs(q) = \bs^{i}\up{i} + \frac{1}{2}\bs^{ij}\up{i}\up{j}
\end{equation}

Define the following: (not correct but mechanically useful)
\begin{equation}
    f = \contraction[1ex]{}{\up{b}}{}{\down{b}}\up{b}\down{b}
\qquad
    \bar{f} = \contraction[1ex]{}{\down{b}}{}{\up{b}}\down{b}\up{b}
\qquad
    (
    \contraction[1ex]{}{\down{b}}{}{\up{b}}\down{b}\up{b}
    -
    \contraction[1ex]{}{\up{b}}{}{\down{b}}\up{b}\down{b}
    )
    = (\bar{f} - f) = 1
\qquad
    \bar{f} = 1, f = 0
\end{equation}
and
\begin{equation}
    \bt_{i} \equiv f\bs_{i}
\qquad
    \bt^{i} \equiv \bar{f}\bs^{i}
\qquad
    \bt^{i}_{j} \equiv \bar{f}f\bs^{i}_{j}
\qquad
    \bt^{ij} \equiv \bar{f}^2\bs^{ij}
\qquad
    \bt_{ij} \equiv f^2\bs_{ij}
\qquad
    \text{and so forth ....}
\end{equation}

The amplitude equation is
\begin{align}
    LHS &= RHS
\\
    \mel{a}{\hat{\Omega}_{\lambda}\dv{\tau}e^{\hat{S}} + \hat{\Omega}_{\lambda}e^{\hat{S}}\varepsilon}{b}
    &= \mel{a}{\hat{\Omega}_{\lambda}\hat{H}e^{\hat{S}}}{b}.
\end{align}
\clearpage

%
%
%
%
%
%


%
%
%
%
%

\begin{align}\begin{split}
    \hat{\Omega} = 1
\\ LHS &=
    i\left(\varepsilon\right)
\\ RHS &=
%
%
(\bh_0 + \bh^{\blue{}\red{}}_{\blue{i}\red{}}\bt^{\blue{i}\red{}} + \frac{1}{2!}\bh^{\blue{}\red{}}_{\blue{ij}\red{}}\bt^{\blue{ij}\red{}} + \frac{1}{2!2!}\bh^{\blue{}\red{}}_{\blue{ij}\red{}}\bt^{\blue{i}\red{}}\bt^{\blue{j}\red{}}) + () + ()
\end{split}\end{align}

%
%
%
%
%

\subsection{Linear Equations}

\begin{align}\begin{split}
    \hat{\Omega} = \down{i}
\\ LHS &=
    i\left(\dv{\bt^{i}_{}}{\tau} + \bt^{i}_{}\varepsilon\right)
\\ RHS &=
%
%
(\bar{f}\bh^{\blue{}\red{z}}_{\blue{}\red{}} + \bh^{\blue{}\red{}}_{\blue{i}\red{}}\bt^{\blue{i}\red{z}} + \bar{f}\bh^{\blue{}\red{z}}_{\blue{i}\red{}}\bt^{\blue{i}\red{}} + \frac{1}{2!}\bh^{\blue{}\red{}}_{\blue{ij}\red{}}\bt^{\blue{i}\red{}}\bt^{\blue{j}\red{z}})
%
%
\\  &+
%
%
    \textit{no linked disconnected terms}
%
%
\\  &+
%
%
(\bh_0 + \frac{1}{2!}\bh^{\blue{}\red{}}_{\blue{i}\red{}}\bt^{\blue{i}\red{}} + \frac{1}{2!2!}\bh^{\blue{}\red{}}_{\blue{ij}\red{}}\bt^{\blue{ij}\red{}})\bt^{\blue{}\red{z}}
\end{split}\end{align}

%
%
%
%
%

\begin{align}\begin{split}
    \hat{\Omega} = \up{i}
\\ LHS &=
    i\left(\dv{\bt^{}_{i}}{\tau} + \bt^{}_{i}\varepsilon\right)
\\ RHS &=
%
%
(f\bh^{\blue{}\red{}}_{\blue{}\red{z}} + f\frac{1}{2!}\bh^{\blue{}\red{}}_{\blue{i}\red{z}}\bt^{\blue{i}\red{}})
%
%
\\  &+
%
%
    \textit{no linked disconnected terms}
%
%
\\  &+
%
%
    \textit{no unlinked disconnected terms}
\end{split}\end{align}

%
%
%
%
%

\newpage
\subsection{Quadratic Equations}

\begin{align}\begin{split}
    \hat{\Omega} = \down{i}\down{j}
\\ LHS &=
    i\left(\dv{\bt^{ij}_{}}{\tau} + \dv{\bt^{i}_{}}{\tau}\bt^{j}_{} + \bt^{i}_{}\dv{\bt^{j}_{}}{\tau} + \bt^{ij}_{}\varepsilon + \bt^{i}_{}\bt^{j}_{}\varepsilon\right)
\\ RHS &=
%
%
(\bar{f}^{2}\bh^{\blue{}\red{zy}}_{\blue{}\red{}} + \bar{f}\bh^{\blue{}\red{z}}_{\blue{i}\red{}}\bt^{\blue{i}\red{y}} + \frac{1}{2!2!}\bh^{\blue{}\red{}}_{\blue{ij}\red{}}\bt^{\blue{i}\red{z}}\bt^{\blue{j}\red{y}})
%
%
\\  &+
%
%
(\frac{1}{2!}\bh_0)\bt^{\blue{}\red{z}}\bt^{\blue{}\red{y}}
    \\  &+  % split long equation
(\bar{f}\bh^{\blue{}\red{y}}_{\blue{}\red{}} + \frac{1}{2!}\bh^{\blue{}\red{}}_{\blue{i}\red{}}\bt^{\blue{i}\red{y}} + \bar{f}\frac{1}{2!}\bh^{\blue{}\red{y}}_{\blue{i}\red{}}\bt^{\blue{i}\red{}})\bt^{\blue{}\red{z}}
%
%
\\  &+
%
%
(\bh_0 + \frac{1}{2!}\bh^{\blue{}\red{}}_{\blue{i}\red{}}\bt^{\blue{i}\red{}} + \frac{1}{2!2!}\bh^{\blue{}\red{}}_{\blue{ij}\red{}}\bt^{\blue{ij}\red{}})\bt^{\blue{}\red{zy}}
\end{split}\end{align}

%
%
%
%
%

\begin{align}\begin{split}
    \hat{\Omega} = \up{i}\down{j}
\\ LHS &=
    i\left(\dv{\bt^{i}_{j}}{\tau} + \dv{\bt^{i}_{}}{\tau}\bt^{}_{j} + \bt^{i}_{}\dv{\bt^{}_{j}}{\tau} + \bt^{i}_{j}\varepsilon + \bt^{i}_{}\bt^{}_{j}\varepsilon\right)
\\ RHS &=
%
%
(f\bar{f}\bh^{\blue{}\red{z}}_{\blue{}\red{y}} + f\frac{1}{2!}\bh^{\blue{}\red{}}_{\blue{i}\red{z}}\bt^{\blue{i}\red{y}})
%
%
\\  &+
%
%
(f\bh^{\blue{}\red{}}_{\blue{}\red{y}} + f\frac{1}{2!2!}\bh^{\blue{}\red{}}_{\blue{i}\red{y}}\bt^{\blue{i}\red{}})\bt^{\blue{}\red{z}}
%
%
\\  &+
%
%
    \textit{no unlinked disconnected terms}
\end{split}\end{align}

%
%
%
%
%

\begin{align}\begin{split}
    \hat{\Omega} = \up{i}\up{j}
\\ LHS &=
    i\left(\dv{\bt^{}_{ij}}{\tau} + \dv{\bt^{}_{i}}{\tau}\bt^{}_{j} + \bt^{}_{i}\dv{\bt^{}_{j}}{\tau} + \bt^{}_{ij}\varepsilon + \bt^{}_{i}\bt^{}_{j}\varepsilon\right)
\\ RHS &=
%
%
(f^{2}\bh^{\blue{}\red{}}_{\blue{}\red{zy}})
%
%
\\  &+
%
%
    \textit{no linked disconnected terms}
%
%
\\  &+
%
%
    \textit{no unlinked disconnected terms}
\end{split}\end{align}

\end{document}